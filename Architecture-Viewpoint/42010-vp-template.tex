\documentclass[10pt]{article}

% The Preamble has many definitions and settings
%%% Preamble
\usepackage{amsmath}
\usepackage{amssymb}
\usepackage{latexsym}
\usepackage{graphicx}


%%%
%%% Bibliography
%%%
\usepackage[backend=bibtex8, citestyle=numeric, bibstyle=authortitle]{biblatex}
\bibliography{/Users/rh/refs/preamble.bib,
  /Users/rh/refs/ieee-1471.bib, /Users/rh/refs/cs.bib, /Users/rh/refs/arch.bib}

%%% 
%%% Definitions
%%%
\newcommand{\Fillin}[1]{\textcolor{Red}{$<$#1$>$}}
\newcommand{\HRule}{\rule{\linewidth}{0.5mm}}
\newcommand{\Optional}{\textcolor{Gray}{\textsf(optional)}}
\newcommand{\angles}[1]{$\langle$#1$\rangle$}
\newcommand{\must}[1]{\textcolor{NavyBlue}{$\star$ #1}}
\newcommand{\note}[1]{\small\textsc{note: }\textit{#1}}
\newcommand{\should}[1]{\textcolor{Plum}{$\Box$ #1}}
\newcommand{\std}[1]{\textcolor{Maroon}{ISO/IEC/IEEE~42010,~#1}}
\newcommand{\tbd}[1]{\noindent\textcolor{Red}{\textbf{TBD: }{\textsf{#1}}}}
\newcommand{\working}[1]{\noindent\textcolor{CadetBlue}{\textsf{#1}}}

% \renewcommand{\thesection}{\Alph{section}}
% \renewcommand{\thesubsection}{\alph{subsection}}
% \renewcommand{\thesubsubsection}{\roman{subsection}.}

\parindent 0cm
\parskip 0.3cm

% \topmargin 0.2cm
% \oddsidemargin 1cm
% \evensidemargin 0.5cm
% \textwidth 14cm
% \textheight 20cm
% \pagestyle{fancy}

%%%  Prettier Tables
% \usepackage{booktabs}
%%%  Diagrams
% \usepackage[all]{xy}


%%%
%%%  Colors
%%%
\usepackage{color}
\usepackage[usenames,dvipsnames]{xcolor} % see: http://en.wikibooks.org/wiki/LaTeX/Colors


%%% Fonts
%\usepackage[charter]{mathdesign}
%\usepackage{arev}
%\usepackage{ccfonts,eulervm} \usepackage[T1]{fontenc}
%\usepackage{cmbright}
%\usepackage{concrete}
%\usepackage{kmath,kerkis}
%\usepackage{mathpazo}
%\usepackage{mathptmx}
\usepackage{newcent}
%\usepackage{palatino}
%\usepackage{pxfonts}
%%% end Fonts


%%% Index
%%% \makeindex


%%% Make this last package (always load after biblatex):
\usepackage[pdfstartview=FitH,colorlinks=true,citecolor=ForestGreen,linkcolor=NavyBlue]{hyperref}

%%% end Preamble




\begin{document}

\title{Architecture Viewpoint Template for ISO/IEC/IEEE 42010}

\author{Rich Hilliard \\
  \href{mailto:r.hilliard@computer.org}{r.hilliard@computer.org}}

\date{\textsc{version} $2.2$}

\maketitle

\pdfinfo{
  /Title (Architecture Viewpoint Template (ISO/IEC/IEEE 42010:2011)
  /Author (Rich Hilliard, r.hilliard@computer.org)
  /Keywords (architecture description, viewpoints)
}


%%%%%%%%%%%%%%%
\begin{abstract}
  \noindent This is a template for specifying architecture viewpoints in
  accordance with ISO/IEC/IEEE~42010:2011, \textit{Systems and
    software engineering---Architecture description}.
\end{abstract}



%%%%%%%%%%%%%%%%%%%%%%%%%%%%%%%%%
\section*{Using the template}
%%%%%%%%%%%%%%%%%%%%%%%%%%%%%%%%%
\addcontentsline{toc}{section}{Using the template}

This is a template that architects and organizations can use for
documenting an architecture viewpoint in accordance with
ISO/IEC/IEEE 42010:2011~\cite{ISO42010:2011}. In particular, the
requirements on viewpoints are found in Clause~7 of that Standard.

The template provides an outline for an architecture viewpoint and
defines a set of ``slots'' or information items to be elaborated by
the architect using the template to define and specify a viewpoint.
Each slot is identified by a heading name followed by a brief
description of its intended content and guidance for developing that
content. 

The template uses a few conventions, as follows.

\must{``Musts'' are items which must be present to satisfy the
  Standard. Musts are marked like this.}

\should{``Shoulds'' are items recommended to be present, but not
  required by the Standard. Shoulds are marked like this.}

Optional items are marked with this: \Optional.

\Fillin{Items} like \Fillin{this} signal names to be filled-in by a
user of the template and used throughout the viewpoint definition.

The source files for this version of the template is packaged as a \LaTeX\
\texttt{section}. It is designed to be included in a \LaTeX\ document
such as those using the report, book or article document class. (Other
versions for Microsoft Word and XML will also be made available.)


\subsection*{License}
\addcontentsline{toc}{subsection}{License}

The \textit{Architecture Viewpoint Template} is copyright \copyright\
2012--2014 by \href{http://www.iso-architecture.org/42010/templates/}%
{Rich Hilliard}.

The latest version is always available at
\url{http://www.iso-architecture.org/42010/templates/}

The template is licensed under a
Creative Commons Attribution 3.0 Unported License. The terms of use
are here: \\
\url{http://creativecommons.org/licenses/by/3.0/}

\vskip .242in
\pdfimage {88x31.png}

This license gives you the user the right to share and remix this
work to define new architecture viewpoints. 
It does not require you to share the results of your usage (i.e., new
viewpoint definitions), but if your use is non-proprietary, we
encourage you to share your viewpoint definition with others for their
use via the WG42 Viewpoint Repository \\
\url{http://www.iso-architecture.org/viewpoints/}.


\subsection*{Version History}
\addcontentsline{toc}{subsection}{Version History}

This template is based on one originally designed for use with
IEEE~std~1471:2000~\cite{IEEE1471:2000} which was published as
\cite{Hilliard:2001b}.  That template formed the basis for the
viewpoint template in \cite{Hilliard:2011}, which appeared during the
development ISO/IEC/IEEE~42010:2011 and subsequently appeared in
Annex~B of the published ISO/IEC/IEEE~42010:2011.

The present document is an enhanced version of these earlier
templates, with additional guidance, clarifications and examples for
readers.
\begin{description}

\item[rev 2.2] 7 October 2014, Moved bibliography from
  \texttt{bibtex} to \texttt{biblatex}. Released revision with minor
  formatting fixes.

\item[rev 2.1b] June 2012, initial release on 42010 website.

\end{description}

\subsection*{Comments or Questions}
\addcontentsline{toc}{subsection}{Comments}
 
Contact the author \href{mailto:r.hilliard@computer.org}%
{Rich Hilliard [r.hilliard@computer.org]} for questions or comments.
For more information on ISO/IEC/IEEE~42010, visit the ISO/IEC/IEEE
42010 website: \\
\url{http://www.iso-architecture.org/42010/}.

%%%%%%%%%%
\vskip .422in
\textcolor{Orange}{\Large The template begins here . . .}

%%% Version 2.1b %%%

%%%%%%%%%%
\section{\Fillin{Viewpoint Name}}\label{vp:template}
%%%%%%%%%%

\must{Provide the name for the viewpoint.}

If there are any synonyms or other common names by which this viewpoint is
known or used, record them here.


%%%%%%%%%%
\section{Overview} 
%%%%%%%%%%

Provide an abstract or brief overview of the viewpoint. 

Describe the viewpoint's key features.


%%%%%%%%%%
\section{Concerns and stakeholders} 
%%%%%%%%%%

Architects looking for an architecture viewpoint suitable for their
purposes often use the identified concerns and typical stakeholders to
guide them in their search.  Therefore it is important (and required
by the Standard) to document the concerns and stakeholders for which a
viewpoint is intended.

%%%%%%%%%%
\subsection{Concerns}\label{vp:concerns}
%%%%%%%%%%

\must{Provide a listing of architecture-relevant concerns to be framed by
this architecture viewpoint per \std{7a}.}

Describe each concern.

Concerns name ``areas of interest'' in a system.

\note{Following ISO/IEC/IEEE 42010, \textbf{system} is a shorthand for
  any number of things including man-made systems, software products
  and services, and software-intensive systems such as ``individual
  applications, systems in the traditional sense, subsystems, systems
  of systems, product lines, product families, whole enterprises, and
  other aggregations of interest''.}

Concerns may be very general (e.g., \textit{Reliability}) or quite
specific (\textit{e.g., How does the system handle network latency?}).
  
Concerns identified in this section are critical information for an
architect because they help her decide when this viewpoint will be
useful.

When used in an architecture description, the viewpoint becomes a
``contract'' between the architect and stakeholders that these
concerns will be addressed in the view resulting from this viewpoint.

It can be helpful to express concerns \emph{in the form of questions}
that views resulting from that viewpoint will be able to answer. E.g.,
\begin{itemize}
\item \textit{How does the system manage faults?}
\item \textit{What services does the system provide?}
\end{itemize}

\note{``In the form of a question'' is inspired by the television quiz
  show, \textit{Jeopardy!}}
 
\std{5.3} contains a candidate list of concerns that must be considered
when producing an architecture description. These can be considered
here for their relevance to the viewpoint being specified:
\begin{itemize}
\item What are the purpose(s) of the system-of-interest?
\item What is the suitability of the architecture for achieving the
  system-of-interest's purpose(s)?
\item How feasible is it to construct and deploy the
  system-of-interest?
\item What are the potential risks and impacts of the
  system-of-interest to its stakeholders throughout its life cycle?
\item How is the system-of-interest to be maintained and evolved?
\end{itemize}

See also: \std{4.2.3}.

%%%%%%%%%%
\subsection{Typical stakeholders} 
%%%%%%%%%%

\must{Provide a listing of the typical stakeholders of a system who
  are in the potential audience for views of this kind, per \std{7b}.}

Typical stakeholders would include those likely to read such views
and/or those who need to use the results of this view for another
task.

Stakeholders to consider include:
\begin{itemize}
\item users of a system; 
\item operators of a system; 
\item acquirers of a system;
\item owners of a system; 
\item suppliers of a system; 
\item developers of a system; 
\item builders of a system; 
\item maintainers of a system.
\end{itemize}

%%%%%%%%%%
\subsection{``Anti-concerns'' \Optional} 
%%%%%%%%%%

It may be helpful to architects and stakeholders to
document the kinds of issues for which this viewpoint is \emph{not
  appropriate or not particularly useful}.

Identifying the ``anti-concerns'' of a given notation or approach may
be a good antidote for certain overly used models and notations.

% \tbd{Examples!}



%%%%%%%%%%
\section{Model kinds+}\label{mk:list}
%%%%%%%%%%

\must{Identify each model kind used in the viewpoint per \std{7c}.}

In the Standard, each architecture view consists of multiple
architecture models. Each model is governed by a \textit{model kind}
which establishes the notations, conventions and rules for models of
that type.  See: \std{4.2.5, 5.5 and 5.6}.

Repeat the next section for each model kind listed here the viewpoint
being specified.


%%%%%%%%%%
\section{\Fillin{Model Kind Name}}\label{vp:mk}
%%%%%%%%%%

\must{Identify the model kind.}


%%%%%%%%%%
\subsection{\Fillin{Model Kind Name} conventions} 
%%%%%%%%%%

\must{Describe the conventions for models of this kind.}

Conventions include languages, notations, modeling techniques,
analytical methods and other operations. These are key modeling
resources that the model kind makes available to architects and
determine the vocabularies for constructing models of the kind and
therefore, how those models are interpreted and used.

It can be useful to separate these conventions into a \emph{language
  part}: in terms of a metamodel or specification of notation to be
used and a \emph{process part}: to describe modeling techniques used
to create the models and methods which can be used on the models that
result.  These include operations on models of the model kind.

The remainder of this section focuses on the language part. The next
section focuses on the process part.

The Standard does not prescribe \emph{how} modeling conventions are to
be documented.  The conventions could be defined:
\begin{description}
\item[I)] by reference to an existing notation or language (such as
  SADT, UML or an architecture description language such as ArchiMate
  or SysML) or to an existing technique (such as $M/M/4$ queues);
\item[II)] by presenting a metamodel defining its core constructs;
\item[III)] via a template for users to fill in;
\item[IV)] by some combination of these methods or in some other
  manner.
\end{description}

Further guidance on methods I) through III) is provided below.
 
Sometimes conventions are applicable across more than one model kind
-- it is not necessary to provide a separate set of conventions, a
metamodel, notations, or operations for each, when a single
specification is adequate.


%%%%%%%%%%
\subsubsection{I) Model kind languages or notations \Optional}
%%%%%%%%%%

Identify or define the notation used in models of the kind.

Identify an existing notation or model language or define one that can
be used for models of this model kind. Describe its syntax, semantics,
tool support, as needed.


%%%%%%%%%%
\subsubsection{II) Model kind metamodel \Optional} 
%%%%%%%%%%

A metamodel presents the AD elements that constitute the
vocabulary of a model kind, and their rules of combination. There are
different ways of representing metamodels (such as UML class diagrams, OWL,
eCore). The metamodel should present:
\begin{description}
\item[entities] What are the major sorts of conceptual elements that
  are present in models of this kind?
\item[attributes] What properties do entities possess in models of
  this kind?
\item[relationships] What relations are defined among entities in
  models of this kind?
\item[constraints] What constraints are there on entities, attributes
  and/or relationships and their combinations in models of this kind?
\end{description}

\note{Metamodel constraints should not be confused with architecture
  constraints that apply to the subject being modeled, not the
  notations used.}

In the terms of the Standard, entities, attributes, relationships are
\textit{AD elements} per \std{3.4, 4.2.5 and 5.7}.

In the \textit{Views-and-Beyond} approach~\cite{DSA:2010}, each
viewtype (which is similar to a viewpoint) is specified by a set of
elements, properties, and relations (which correspond to entities,
attributes and relationships here, respectively).

When a viewpoint specifies multiple model kinds it can be useful to
specify a single viewpoint metamodel unifying the definition of the
model kinds and the expression of correspondence rules.  When defining
an architecture framework, it may be helpful to use a single metamodel
to express multiple, related viewpoints and model kinds.

% \tbd{EXAMPLE -- In \cite{Hilliard:1999} and earlier work, we said that
%   all views are built from primitives called components, connections
%   and constraints which basically gives views a graph structure with
%   components as nodes and two types of edges (connections and
%   constraints). There are two issues with this: (\textit{1})
%   components and \textit{connectors} have taken on a specialized
%   meaning from the work by CMU and others \cite{Shaw-Garlan:1996};
%   (\textit{2}) this ur-ontology may be over-commiting for some views.}


%%%%%%%%%%
\subsubsection{III) Model kind templates \Optional}
%%%%%%%%%%

Provide a template or form specifying the format and/or content of
models of this model kind.

%% \tbd{EXAMPLE} 


%%%%%%%%%%
\subsection{\Fillin{Model Kind Name} operations \Optional} 
%%%%%%%%%%

Specify operations defined on models of this kind.

See~\S\ref{Opns} for further guidance.


%%%%%%%%%%
\subsection{\Fillin{Model Kind Name} correspondence rules}
%%%%%%%%%%

\must{Document any correspondence rules associated with the model
  kind.}

See~\S\ref{CRs} for further guidance.


%%%%%%%%%%
\section{Operations on views}\label{Opns}
%%%%%%%%%%

Operations define the methods to be applied to views and their models.
Types of operations include:

\begin{description}

\item[construction methods] are the means by which views are
  constructed under this viewpoint. These operations could be in the
  form of process guidance (how to start, what to do next); or work
  product guidance (templates for views of this type). Construction
  techniques may also be heuristic: identifying styles, patterns, or
  other idioms to apply in the synthesis of the view.

\item[interpretation methods] which guide readers to understanding
  and interpreting architecture views and their models.

\item[analysis methods] are used to check, reason about, transform,
  predict, and evaluate architectural results from this view,
  including operations which refer to model correspondence rules.

\item[implementation methods] are the means by which to design and
  build systems using this view.

\end{description}

Another approach to categorizing operations is from Finkelstein et
al. \cite{Finkelstein+1992}. The \emph{work plan} for a viewpoint
defines 4 kinds of actions (on the view representations):
\textit{assembly actions} which contains the actions available to the
developer to build a specification; \textit{check actions} which
contains the actions available to the developer to check the
consistency of the specification; \textit{viewpoint actions} which
create new viewpoints as development proceeds; \textit{guide actions}
which provide the developer with guidance on what to do and when.


%%%%%%%%%%
\section{Correspondence rules}\label{CRs}
%%%%%%%%%%

\must{Document any correspondence rules defined by this viewpoint or
  its model kinds.}

Usually, these rules will be across models or across views since,
constraints within a model kind will have been specified as part of
the conventions of that model kind.

See: \std{4.2.6 and 5.7}

%%\tbd{examples or specs}

%%%%%%%%%%
\section{Examples \Optional} 
%%%%%%%%%%

Provide helpful examples of use of the viewpoint for the reader
(architects and other stakeholders).


%%%%%%%%%%
\section{Notes \Optional} 
%%%%%%%%%%

Provide any additional information that users of the viewpoint may
need or find helpful.


%%%%%%%%%%
\section{Sources} 
%%%%%%%%%%

\must{Identify sources for this architecture viewpoint, if any,
  including author, history, bibliographic references, prior art, per
  \std{7e}.}



\textcolor{Orange}{\Large The template ends here!}
%%%%%%%%%%


%%%%%%%%%% Bibliography
\printbibliography
%%%%%%%%%%

%%%%%%%%%%
% \include{index}
% \addcontentsline{toc}{chapter}{Index}
%%%%%%%%%%


\newpage
%%%%%%%%%%
\tableofcontents 
%%%%%%%%%%

\end{document}
