\documentclass[a4paper]{article}

\usepackage{fullpage}
\usepackage{parskip}
\usepackage[english]{babel}
\usepackage{amsmath}
\usepackage{amssymb}
\usepackage{graphicx}
\usepackage{indentfirst}
\usepackage{listings}
\usepackage{enumitem}

\begin{document}

\begin{titlepage}

\centering

\vfill

\includegraphics[width=10cm]{Wits-logo1.jpg}

\vskip 0.1cm

\center 

\textsc{\LARGE University of the Witwatersrand}\\[0.5cm] 

\textsc{\Large Software Design \\ Software Documentation Specifications} \\[0.5cm] 

\begin{minipage}{0.4\textwidth}

\begin{center} \large

\textbf{Team Name:} \\[0.3cm]

\textsc{Level Seven Crew} \\[0.3cm]

\end{center}

\begin{center} \large

\textbf{Product Name:} \\[0.3cm]

\textsc{HealFolio} \\[0.3cm]

\end{center}

\begin{flushleft} \large

\textbf{Team Members:} \\[0.3cm]

Jan \textsc{Badenhorst} \\
Adam \textsc{Lerumo} \\
Tumbone \textsc{Asukile} \\
Daniel da \textsc{Silva} \\

\end{flushleft}

\end{minipage} \\[0.7cm]

\begin{minipage}{0.4\textwidth}

\begin{flushright} \large

\textbf{Senior Lecturer:} \\[0.3cm]

Dr. Terence van \textsc{Zyl} \\

\end{flushright}

\end{minipage} \\[2cm]

{\large July 27, 2016}

{\emph{Updated:} August 12, 2016}
    
\end{titlepage}

\setlength\parindent{24pt}

\section{Functional requirements \emph{(``shall lists")}}

\subsection{Medical Practitioner Dashboard \emph{(log in access required)}}

The doctor/nurse should be able to update the patients information who is consulting at that time of consultation. The dashboard will also have the following functionality:

\subsubsection{Patient Profile}

\begin{itemize}

\item \textbf{Name}

\item \textbf{Surname}

\item \textbf{Identification number \emph{(Primary search key)}}

\item \textbf{Address}

\item \textbf{Next of kin contact details}

\item \textbf{Employment status}

\item \textbf{Marital Status}

\item \textbf{Practitioner visited history}

\item \textbf{Payment history}

\item \textbf{Payment method \emph{(eg. medical aid)}} 

\item \textbf{Illnesses}

\item \textbf{Previous Diagnosis}

\item \textbf{Previous prescriptions from other practitioners \emph{(valid for a certain period)}}

\item \textbf{Medication given to patient by pharmacy}

\item \textbf{Allergies}

\end{itemize}

\subsubsection{Practice Profile}

\begin{itemize}

\item \textbf{Doctor / Nurse \emph{(User Authentication - Required)}}

\item \textbf{Name}

\item \textbf{Practice number}

\item \textbf{Address}

\item \textbf{Operating hours}

\item \textbf{Inventory available \emph{(System should be able to alert if suppliers are low)}}

\end{itemize}

\subsection{Systems Admin Dashboard}

It should have all of the functionality mentioned above as well as the following:

\subsubsection{Doctor/Nurse Profile \emph{(Systems Admin authentication)}}

We need this to keep track of which doctor makes changes to the database.

\begin{itemize}

\item \textbf{Name}

\item \textbf{Surname}

\item \textbf{ID number}

\item \textbf{Practice number}

\item \textbf{Contact details}

\item \textbf{Practice history}

\end{itemize}

\section{Nonfunctional requirements}

\subsection{Usability}

\subsection{Reliability}

\subsection{Performance}

\subsection{Support ability}

\subsection{Implementation}

\subsection{Interface}

\subsection{Packaging}

\subsection{Legal}

\section{Input Requirements}

\subsection{Patient identifier key - ID number}

\subsection{Doctor/Nurse identifier key and user access - Practice Number}

\subsection{System identifier key and user access - ID number}

\section{Process Requirements}

\subsection{Database transaction}

The system must be able to send, receive and trigger transaction to database system.

\subsection{Data integrity}

Commit transactions that are completed and/or rollback unfinished or time-out transactions.

\subsection{Data validation}

Data error from the user’s end and from the back-end database-processing end must be gracefully handled. There will be data validation and error-handling routines as part of the patient registration system.

\subsection{Performance}

Must resolve locking issues and must not allow concurrent access to patient information. Send, receive and display user messages to assist the over-all user experience.

\subsection{Data repository}

The  system will maintain the existing patient information on the database as the main repository of data.

\section{Output Requirements}

\subsection{Transaction summary and confirmation}

\subsection{Exception reports}

System exception reports must be consolidated to record special patient records or special conditions not normally handled .Examples are diabetics, high/low blood pateients, epileptic.

\subsection{Registration Reports and summaries}

System administrator and Doctors/nurses must be able to extract summarized and rolled-up data into meaningful information. All records will be archived but accessible on demand.

\section{Software Requirements}

\subsection{Client Operating Systems}

Windows, Linux
	
\end{document}