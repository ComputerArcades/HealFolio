\documentclass[a4paper]{article}

\usepackage{fullpage}
\usepackage{parskip}
\usepackage[english]{babel}
\usepackage{amsmath}
\usepackage{amssymb}
\usepackage{graphicx}
\usepackage{indentfirst}
\usepackage{listings}
\usepackage{enumitem}
\usepackage{hyperref}

\hypersetup{colorlinks = true, linkcolor = blue, filecolor = magenta, urlcolor = cyan}
 
\urlstyle{same}

\begin{document}

\begin{titlepage}

\centering

\vfill

\includegraphics[width=10cm]{Wits-logo1.jpg}

\vskip 0.1cm

\center 

\textsc{\LARGE University of the Witwatersrand}\\[0.5cm] 

\textsc{\Large Software Design Proposal} \\[0.5cm] 

\begin{minipage}{0.4\textwidth}

\begin{center} \large

\textbf{Team Name:} \\[0.3cm]

\textsc{Level Seven Crew} \\[0.3cm]

\end{center}

\begin{center} \large

\textbf{Product Name:} \\[0.3cm]

\textsc{HealFolio} \\[0.3cm]

\end{center}

\begin{flushleft} \large

\textbf{Team Members:} \\[0.3cm]

Tumbone Asukile\newline
Adam Lerumo\newline
Daniel da Silva\newline
Jan Badenhorst\newline

\end{flushleft}

\end{minipage} \\[0.7cm]

\begin{minipage}{0.4\textwidth}

\begin{flushright} \large

\textbf{Senior Lecturer:} \\[0.3cm]

Dr. Terence Van Zyl

\end{flushright}

\end{minipage} \\[1cm]

{\large Date\_here} 
    
\end{titlepage}

\setlength\parindent{24pt}

\newpage

\section{Introduction}

\subsection{\textit{Purpose of the System}}

The software \emph{(HealFolio)} is aimed at improving service delivery for all types of medical practitioners; specifically the filing system currently used by medical practitioners. This software will provide doctors with easier access to patient records and can provide more security. It will also help them make more accurate diagnoses, leading to better prescriptions and more, when the software is further developed. 
		
\subsection{\textit{Scope of the System}}

The software will be use by doctors, assistants and even patients themselves. By using a web based application the each of the following users of the software can easily get to and use the information by accessing the user friendly interface. This will make it possible for both the service provider and receiver to give accurate diagnosis and critical feedback

\subsection{\textit{Objectives and Success Criteria of the Project}}

\emph{Objectives:}

\begin{itemize} 

\item To make the precious minutes count in an emergency and avoid any long term complications.

\item Take away the unnecessary revisiting or come backs to doctors to a minimal.

\item Get accurate diagnosis at the first visit for recurring problems.

\item Taking the admin work away for patients, especially for the elderly.

\end{itemize}

\emph{Success:}

\begin{itemize} 

\item Making the software easy to use and understand.

\item Open to adaptations and future upgrades.

\item Robust and workable on most platforms. \emph{(Mobile friendly, Linux)}

\item Having the system to be alert in failure to connect to server, which makes it virtually impossible to loose information. 

\end{itemize}
		
\subsection{\textit{References}}

\href{ https://cle.wits.ac.za/access/content/group/COMS3009_2016/book-SE_marsic.pdf}{Software Engineering}
		
\subsection{\textit{Overview}}

Utilizing the resources available and time to create the optimal running application that will incorporate every feature desired by the clients. Keeping in mind future growth in user numbers by keeping efficiency to the maximum.

\newpage

\section{Current System}
	
\newpage

\section{Proposed system}

\subsection{\textit{Overview}}
		
\subsection{\textit{Functional Requirements}}
		
\subsection{\textit{Nonfunctional Requirements}}
		
\subsection{\textit{System models}}

\newpage

\section{Glossary}

\end{document}
