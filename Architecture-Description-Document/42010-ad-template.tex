\documentclass[10pt,oneside]{report}

\usepackage{indentfirst}
\usepackage{placeins}

% The preamble contains definitions and settings
\input{preamble}

\begin{document}

\begin{titlepage}

\centering

\includegraphics[width=7cm]{Wits-logo1.jpg}

\vskip 0.1cm

\center 

\textsc{\LARGE University of the Witwatersrand}\\[0.4cm] 

\textsc{\Large Architecture Description of \\ 
    Layered Architecture for \\[0.1cm] 
    HealFolio} \\[0.4cm] 

\begin{minipage}{0.4\textwidth}

\begin{center} \large

\textbf{Team Name:} \\[0.2cm]

\textsc{Level Seven Crew} \\[0.2cm]

\end{center}

\begin{center} \large

\textbf{Product Name:} \\[0.2cm]

\textsc{HealFolio} \\[0.7cm]

\end{center}

\begin{flushleft} \large

\textbf{Team Members:} \\[0.2cm]

Jan \textsc{Badenhorst} \\
Adam \textsc{Lerumo} \\
Tumbone \textsc{Asukile} \\
Daniel da \textsc{Silva} \\

\end{flushleft}

\end{minipage} \\[0.5cm]

\begin{minipage}{0.4\textwidth}

\begin{flushright} \large

\textbf{Senior Lecturer:} \\[0.2cm]

Dr. Terence van \textsc{Zyl} \\

\end{flushright}

\end{minipage} \\[0.5cm]

{\large September 12, 2016} 
    
\end{titlepage}

\newpage
%%%%%%%%%%
\tableofcontents
% \listoffigures
% \listoftables
%%%%%%%%%%

%%%
%%% Change page number for main body of text
%%%
\pagenumbering{arabic}

\setlength\parindent{24pt}

%%%%%%%%%%
\chapter{Introduction}\label{ad:info}
%%%%%%%%%%

This chapter describes introductory information items of the Architecture Description (AD), including identifying and supplementary information.

%%%%%%%%%%
\section{Identifying information}\label{ad:idinfo}
%%%%%%%%%%

\begin{itemize}
\item The architecture being expressed is an Layered Architecture.
\item The system of interest is HealFolio, an individual applications for which this is an architecture description.
\end{itemize}

Following ISO/IEC/IEEE 42010, \textit{system} (or \textit{system-of-interest}) is a shorthand for any number of things including man-made systems, software products and services, and software-intensive systems including ``individual applications, systems in the traditional sense, subsystems, systems of systems, product lines, product families, whole enterprises, and other aggregations of interest''. \std{4.2}

%%%%%%%%%%
\section{Supplementary information}\label{ad:supinfo}
%%%%%%%%%%

\subsection{Date of issue and status}

\begin{itemize}
\item This project was started on the 24 of July 2016.
\end{itemize}

\subsection{Authors}

\begin{itemize}
\item Adam Lerumo - Product owner
\item Tumbone Asukile - Developer
\item Daniel da Silva - Developer
\item Jan Badenhorst - Scrum Master
\end{itemize}

\subsection{Reviewers}

\begin{itemize}
\item Dr. Terence van Zyl - Senior Lecturer and intellectual guide.
\end{itemize}

\subsection{Scope}

The team will be designing software (HealFolio) aimed at improving service delivery for all types of medical practitioners; specifically the filing system currently used by medical practitioners. This software will provide doctors with easier access to patient records and provide more security. It will also help them make more accurate diagnoses, provide better prescriptions and more as the software is further developed.

\subsection{Context}


The software to be implemented will operate in the medical industry, predominantly among General Practitioners (GP's). The issues in this sector include things such as (but not limited to):

\begin{itemize}

\item Patient record sharing across many different practices.

\item Security in the use of medical aid for visits to GP's and other practitioners.

\item Patient reaction to prescribed medications.

\end{itemize}

The above list provides a brief outline of the core issues to be addressed by the project, more issues may be identified as the project progresses.

\subsection{Overview}

The Layered Architecture pattern the is being used in HealFolio, otherwise known as the n-tier architecture pattern. This pattern is the standard for most Java EE applications and therefore is widely known by most architects, designers, and developers. The layered architecture pattern closely matches the traditional IT communication and organizational structures found in most companies, making it a natural choice for most business application development efforts.

\begin{figure}[h!]
\includegraphics[width=15cm]{Layered-architecture-pattern.png}
\caption{Layered Architecture Pattern}
\label{Layered Architecture Pattern}
\end{figure}

This project is leaning toward the Microservices Architecture pattern. There are several common core concepts that apply to this architecture pattern. The first of these concepts is the notion of separately deployed units. Each component of the microservices architecture is deployed as a separate unit, allowing for easier deployment through an effective and streamlined delivery pipeline, increased scalability, and a high degree of application and component decoupling within your application. Perhaps the most important concept to understand with this pattern is the notion of a service component. Rather than think about services within a microservices architecture, it is better to think about service components, which can vary in granularity from a single module to a large portion of the application. Service components contain one or more modules (e.g., Java classes) that represent either single-purpose function (e.g., viewing information for a specific patient) or an independent portion of a our application (e.g., signing up for the service).

\begin{figure}[h!]
\includegraphics[width=15cm]{Basic-Microservices-architecture-pattern.png}
\caption{Basic Microservices architecture pattern}
\label{Basic Microservices architecture pattern}
\end{figure}

%%%%%%%%%%
\subsection{Architecture evaluations}
%%%%%%%%%%

Components within the layered architecture pattern are organized into horizontal layers, each layer performing a specific role within the application (e.g., presentation logic or business logic). Although the layered architecture pattern does not specify the number and types of layers that must exist in the pattern, most layered architectures consist of four standard layers: presentation, business, persistence, and database. In the case of our application so far the business layer and persistence layer are combined into a single business layer (a controller), particularly when the persistence logic is embedded within the business layer components. Thus our small application has only three layers.

Each layer of the layered architecture pattern has a specific role and responsibility within the application. The presentation layer is responsible for handling all user interface and browser communication logic, whereas as the business layer is responsible for executing specific business rules associated with the request and retrieving of information. Each layer in the architecture forms an abstraction around the work that needs to be done to satisfy a particular business request. For example, the presentation layer doesn’t need to know or worry about how to get customer data; it only needs to display that information on a screen in particular format. The business layer doesn’t need to be concerned about how to format patient data for display on a screen it needs to get the data from the database layer, perform business logic against the data (e.g., finding the diagnostic history of certain patients), and pass that information up to the presentation layer. 

One of the powerful features of the layered architecture pattern is the separation of concerns among components. Components within a specific layer deal only with logic that pertains to that layer. For example, components in the presentation layer deal only with presentation logic, whereas components residing in the business layer deal only with business logic. This type of component classification makes it easy to build effective roles and responsibility models into our architecture, and also makes it easy to develop, test, govern, and maintain applications using this architecture pattern due to well-defined component interfaces and limited component scope. 

%%%%%%%%%%
\subsection{Rationale for key decisions}
%%%%%%%%%%

The layered architecture pattern is a solid general-purpose pattern, making it a good starting point for most applications, particularly when you are not sure what architecture pattern is best suited for your application. However, there are a couple of things to consider from an architecture standpoint when choosing this pattern.

The first thing to watch out for is what is known as the architecture sinkhole anti-pattern. This anti-pattern describes the situation where requests flow through multiple layers of the architecture as simple pass-through processing with little or no logic performed within each layer. Every layered architecture will have at least some scenarios that fall into the architecture sinkhole anti-pattern. The key, however, is to analyze the percentage of requests that fall into this category. If one finds a majority of requests are simple pass-through processing, you might want to consider making some of the architecture layers open (meaning requests are allowed to bypass this open layer and go directly to the layer below it), keeping in mind that it will be more difficult to control change due to the lack of layer isolation.

Another consideration with the layered architecture pattern is that it tends to lend itself toward monolithic applications, even if you split the presentation layer and business layers into separate deploy-able units. While this may not be a concern for some applications, it does pose some potential issues in terms of deployment, general robustness and reliability, performance, and scalability.


%%%%%%%%%%
\chapter{Stakeholders and concerns}\label{sec:snc}
%%%%%%%%%%

This chapter is intended to identify and outline the various stakeholders and concerns associated with the product. It provides a traceability matrix to determine who key choices in the architecture will affect.

%%%%%%%%%%
\section{Stakeholders}\label{ad:stakeholders}
%%%%%%%%%%

The stakeholders are divided into two main categories below. These two categories are divided based on the activity of the stake holders namely using the product and providing the product.
 
 \begin{itemize}

\item \textbf{Customers and end users}
The application will interact with people in the medical field, such as GPs and specialists. Patients are also secondary users in that they generate the information.
\begin{itemize}
	\item \emph{Medical practitioners}
	\item \emph{Patients}
\end{itemize}


\item \textbf{Systems architects and developers}
This section includes people involved in the development, deployment and continuance of the system. We (Level Seven Crew) currently consider ourselves as the development team.
\begin{itemize}
	\item \emph{Organization providing the system}
	\item \emph{System administrator}
	\item \emph{Maintenance team}
	\item \emph{Development team}
\end{itemize}

\end{itemize}

%%%%%%%%%%
\section{Concerns}\label{ad:concerns}
%%%%%%%%%%

This section aims at answering critical questions as an aid to identifying the concerns as listed in 2.3.1.

\begin{itemize}
\item \textbf{What are/is the purpose(s) of the system-of-interest?} \\
Providing (often sensitive) information about patients to doctors allowing them to make well-informed diagnoses/prescriptions etc.

\item \textbf{What is the suitability of the architecture for achieving the system-of-interest's purpose(s)?} \\
The layered architecture is suitable in that it allows easy maintenance for things such as updating business rules, new information requirements etc.

\item \textbf{How feasible is it to construct and deploy the system-of-interest?} \\
Given the time allowance, the open source tools at our disposal and the guidance of Dr. Terence van Zyl, constructing the system is entirely feasible. The deployment would involve paying a hosting service which we consider outside the scope of the course. The application will be hosted locally during development while firebase is hosted for free by google. We are still, however, mindful of providing a system that can be easily deployed.

\item \textbf{What are the potential risks and impacts of the system-of-interest to its stakeholders throughout its life cycle?} \\
There are multiple legal and social implications of a system like this. Most of these are related to security and as such can be remedied with consideration for this. The lists below outline a few of the risk factors involved in the system.
\begin{itemize}
	\item \textbf{Social}
		Some of these may fall out of scope of the system as it is exceedingly difficult to write software 
		\begin{itemize}
			\item Unauthorized entities accessing sensitive information on patients.
			\item Doctors mis-using the system for personal reasons. (eg accessing information on a spouse without permission)
			\item Doctors sharing information with people who the patient did not authorize.
		\end{itemize}
	\item \textbf{Legal}
		These relate to net neutrality laws. This is a key concern in the development of this system, as we may find laws regarding medical practices clashing with net neutrality laws.
		\begin{itemize}
			\item Users need to have access to the data stored about them.
			\item Users need to be able to update data stored about them.
			\item Users need to be able to remove data stored about them.
			\item Users need to be able to determine who views there information.
		\end{itemize}
\end{itemize}

\item \textbf{How is the system-of-interest to be maintained and evolved?} \\
The system needs to be flexible in order to adapt to ever-changing laws regarding net neutrality and privacy. The medical field is also rapidly changing with new diagnostics and treatments being invented all the time, the system needs to be able to provide facilities which would provide user-value should the need arrive. This relates to changing business values and the decision to use a layered architecture model.
\end{itemize}

%%%%%%%%%%
\section{Concern--Stakeholder Trace-ability}
%%%%%%%%%%

\subsection{List of concerns:}

\begin{itemize}
\item \textbf{Concern 1} - Security\\
	Security is a major concern as the information stored is highly sensitive. This refers mainly to cyber attacks with the aim of stealing information.
\item \textbf{Concern 2} - Doctor patient confidentiality\\
	Doctor patient confidentiality ties into security but is the more specific case of the doctor's ability to share patient profiles.
\item \textbf{Concern 3} - Future growth and expansion\\
	The system needs to be able to grow in multiple directions e.g. New functionalities as well as new patient information.
\item \textbf{Concern 4} - Doctors mis-using the information\\
	Doctors should not be allowed to look up patients' information where the patient hasn't approved.
\item \textbf{Concern 5} - Network latency and loading times\\
	The doctor should not have to wait exorbitant amounts of time for loading, consults generally last maximum 15 minutes.
\item \textbf{Concern 6} - Ease of use\\
	The layout should be logical and easy to navigate.
\item \textbf{Concern 7} - Deployment\\
	Deployment needs to be done in a way that provides access to multiple areas in South Africa to reduce instances of missing information.
\item \textbf{Concern 8} - Changing business rules\\
	Ties in to Future growth and expansion.
\item \textbf{Concern 9} - Debugging\\
	The product needs to be able to be debugged with a minimal amount of effort.
\item \textbf{Concern 10} - Net neutrality\\
	Users need to be able to access, update and remove their information at will.
\end{itemize}

\subsection{List of Stakeholders}

\begin{itemize}
\item \textbf{Stakeholder 1} - Medical practitioners
\item \textbf{Stakeholder 2} - Patients
\item \textbf{Stakeholder 3} - Organization providing the system
\item \textbf{Stakeholder 4} - System administrator
\item \textbf{Stakeholder 5} - Maintenance team
\item \textbf{Stakeholder 6} - Development team
\end{itemize}

\begin{table}[h]
\caption{Showing association of stakeholders to concerns in an AD}\label{stakeholders concerns}
\begin{tabular}{ l | c | c | c | c | c | c }
				&\textsf{S.holder~1}	&\textsf{S.holder~2}	&\textsf{S.holder~3}	&\textsf{S.holder~4}	&\textsf{S.holder~5}	&\textsf{S.holder~6}	\\
\hline
\textsf{Concern~1}	&	\textbf{X}		&	\textbf{X}		&	\textbf{X}		& 				&				&				\\
\textsf{Concern~2}	&				&	\textbf{X}		& 				& 				&				&				\\
\textsf{Concern~3}	&				&				&	\textbf{X}		& 	\textbf{X}		&	\textbf{X}		&				\\
\textsf{Concern~4}	&				&	\textbf{X}		&	\textbf{X}		& 				&				&				\\
\textsf{Concern~5}	&	\textbf{X}		&	\textbf{X}		& 				& 				&				&				\\
\textsf{Concern~6}	&	\textbf{X}		&	\textbf{X}		& 				& 				&				&				\\
\textsf{Concern~7}	&	\textbf{X}		&	\textbf{X}		&	\textbf{X}		&	\textbf{X}		&				&				\\
\textsf{Concern~8}	&				& 				&	\textbf{X}		& 	\textbf{X}		&	\textbf{X}		&				\\
\textsf{Concern~9}	&				&				& 				& 				&	\textbf{X}		&	\textbf{X}		\\
\textsf{Concern~10}	&	\textbf{X}		&	\textbf{X}		&	\textbf{X}		& 				&				&				\\
\end{tabular}
\end{table}

%%%%%%%%%%
\chapter{Viewpoints}\label{ad:vps}
%%%%%%%%%%

\textbf{Definition}: A viewpoint is a collection of patterns, templates, and conventions for constructing one type of view. It defines the stakeholders whose concerns are reflected in the viewpoint and the guidelines, principles, and template models for constructing its views.

%%%%%%%%%%
\section{Architecture Model}\label{mk:list}
%%%%%%%%%%

A model is a complete, basic, and simplified description of software architecture which is composed of multiple views from a particular perspective or viewpoint.

A view is a representation of an entire system from the perspective of a related set of concerns. It is used to describe the system from the viewpoint of different stakeholders such as end-users, developers, project managers, and testers.


%%%%%%%%%%
\subsection{4+1 Model}\label{vp:mk}
%%%%%%%%%%

The 4+1 View Model was designed by Philippe Kruchten to describe the architecture of a software-intensive system based on the use of multiple and concurrent views. It is a multiple view model that addresses different features and concerns of the system. It standardizes the software design documents and makes the design easy to understand by all stakeholders.

It is an architecture verification method for studying and documenting software architecture design and covers all the aspects of software architecture for all stakeholders. It provides four essential views:

\begin{figure}[h!]
\includegraphics[width=15cm]{four-plus-one-view-model.jpg}
\caption{4+1 Model}
\label{Basic Micro-services architecture pattern}
\end{figure}

\begin{itemize}
\item \textbf{The logical view or conceptual view} - It describes the object model of the design.
\item \textbf{The process view} - It describes the activities of the system, captures the concurrency and synchronization aspects of the design.
\item \textbf{The physical view} - It describes the mapping of software onto hardware and reflects its distributed aspect.
\item \textbf{The development view} - It describes the static organization or structure of the software in its development of environment.
\end{itemize}

This view model can be extended by adding one more view called scenario view or use case view for end-users or customers of software systems. It is coherent with other four views and are utilized to illustrate the architecture serving as “plus one” view, (4+1) view model. The following figure describes the software architecture using five concurrent views (4+1) model.

%%% include Viewpoint Template here:
%%% Version 2.1b %%%

\section{Logical Viewpoint}

\subsection{Overview}

Describes the system’s functional elements, their responsibilities, interfaces, and primary interactions. A Functional view is the cornerstone of most ADs and is often the first part of the description that stakeholders try to read. It drives the shape of other system structures such as the information structure, concurrency structure, deployment structure, and so on. It also has a significant impact on the system’s quality properties such as its ability to change, its ability to be secured, and its run-time performance. 

\subsection{Concerns and stakeholders}

\subsubsection{Concerns} 

\begin{itemize}
\item Functional capabilities. 
\item External interfaces.
\item Internal structure.
\item Functional design philosophy.
\end{itemize}

\subsubsection{Stakeholders}
\begin{itemize}
\item All stakeholders.
\end{itemize}

\subsection{Model}

\begin{itemize}
\item Functional structure model.
\end{itemize}

\subsection{Known issues with view}

\begin{itemize}
\item Poorly defined interfaces.
\item Poorly understood responsibilities. 
\item Infrastructure modeled as functional elements.
\item Overloaded view.
\item Diagrams without element definitions. 
\item Difficulty in reconciling the needs of multiple stakeholders.
\item Wrong level of detail.
\item `God’ elements
\item Too many dependencies.
\end{itemize}

\FloatBarrier

\begin{figure}[h!]
\centering
\caption{Logical Model}
\includegraphics[width=16cm]{Logical.png}
\label{Basic Micro-services architecture pattern}
\end{figure}

\FloatBarrier

\section{Development Viewpoint}

\subsection{Overview}

Describes the architecture that supports the software development process. Development views communicate the aspects of the architecture of interest to those stakeholders involved in building, testing, maintaining, and enhancing the system.

\subsection{Concerns and stakeholders}

\subsubsection{Concerns}

\begin{itemize}
\item Module organization.
\item Common processing.
\item Standardization of design.
\item Standardization of testing.
\item Instrumentation.
\item Code-line organization.
\end{itemize}

\subsubsection{Stakeholders}

\begin{itemize}
\item Production engineers, software developers and testers.
\end{itemize}

\subsection{Model}

\begin{itemize}
\item Module structure models.
\item Common design models.
\item Code-line models.
\end{itemize}

\subsection{Known issues with view}

\begin{itemize}
\item Too much detail.
\item Overburdening the AD.
\item Uneven focus. 
\item Lack of developer focus.
\item Lack of precision.
\item Problems with the specified environment.
\end{itemize}

\FloatBarrier

\begin{figure}[h!]
\centering
\caption{Deployment Model}
\includegraphics[width=16cm]{Deployment.png}
\label{Basic Micro-services architecture pattern}
\end{figure}

\FloatBarrier

\section{Physical Viewpoint}

\subsection{Overview}

Describes the environment into which the system will be deployed, including capturing the dependencies the system has on its run-time environment. This view captures the hardware environment that your system needs (primarily the processing nodes, network interconnections, and disk storage facilities required), the technical environment requirements for each element, and the mapping of the software elements to the run-time environment that will execute them. 

\subsection{Concerns and stakeholders}

\subsubsection{Concerns}

\begin{itemize}
\item Run-time platform required.
\item Specification and quantity of hardware or hosting required.
\item Third-party software requirements.
\item Technology compatibility.
\item Network requirements.
\item Network capacity required.
\item Physical constraints.
\end{itemize}

\subsubsection{Stakeholders}

\begin{itemize}
\item System administrators, developers, testers, communicators, and assessors.
\end{itemize}

\subsection{Model}

\begin{itemize}
\item Run-time platform models.
\item Network models.
\item Technology dependency models.
\item Inter-model relationships.
\end{itemize}

\subsection{Known issues with view}

\begin{itemize}
\item Unclear or inaccurate dependencies.
\item Unproven technology.
\item Unsuitable or missing service-level agreements.
\item Lack of specialist technical knowledge.
\item Late consideration of the deployment environment.
\item Ignoring inter-site complexities.
\item Inappropriate headroom provision.
\item Not specifying a disaster recovery environment.
\end{itemize}

\FloatBarrier

\begin{figure}[h!]
\centering
\caption{Physical Model}
\includegraphics[width=12cm]{Physical_model.png}
\label{Basic Micro-services architecture pattern}
\end{figure}

\FloatBarrier

\section{Process Viewpoint}

\subsection{Overview}

Describes how the system will be operated, administered, and supported when it is running in its production environment. For all but the simplest systems, installing, managing, and operating the system is a significant task that must be considered and planned at design time. The aim of the Operational viewpoint is to identify system-wide strategies for addressing the operational concerns of the system’s stakeholders and to identify solutions that address these.

\subsection{Concerns and stakeholders}

\subsubsection{Concerns}

\begin{itemize}
\item Installation and upgrade.
\item Functional migration. 
\item Data migration.
\item Operational monitoring and control.
\item Alerting.
\item Configuration management.
\item Performance monitoring.
\item Support.
\item Backup and restore.
\item Operation in third-party environments.
\end{itemize}

\subsubsection{Stakeholders}

\begin{itemize}
\item System administrators, developers, testers, communicators, and assessors.
\end{itemize}

\subsection{Model}

\begin{itemize}
\item Installation models.
\item Migration models.
\item Configuration management models. 
\item Administration models.
\item Support models.
\end{itemize}

\subsection{Known issues with view}

\begin{itemize}
\item Lack of engagement with the operational staff.
\item Lack of back-out planning.
\item Lack of migration planning.
\item Insufficient migration window.
\item Missing management tools.
\item Production environment constraints.
\item Lack of integration into the production environment 
\item Inadequate backup models.
\item Unsuitable alerting.
\end{itemize}

\FloatBarrier

\begin{figure}[h!]
\centering
\caption{Process Model}
\includegraphics[width=16cm]{Process.png}
\label{Basic Micro-services architecture pattern}
\end{figure}

\FloatBarrier

%%%%%%%%%% Bibliography
%\printbibliography
%%%%%%%%%%

\end{document}
